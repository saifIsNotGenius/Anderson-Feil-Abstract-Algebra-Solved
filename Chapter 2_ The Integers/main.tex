\documentclass[12pt]{article}
 
\usepackage[margin=1in]{geometry} 
\usepackage{amsmath,amsthm,amssymb}
\usepackage{graphicx}
\usepackage{mathtools}
\usepackage{changepage}
\usepackage{listings}
% \usepackage{minted}
% \setminted{baselinestretch=0.8}
\linespread{1.15}

\newenvironment{exercise}[2][Exercise]{\begin{trivlist}
\item[\hskip \labelsep {\bfseries #1}\hskip \labelsep {\bfseries #2.}]}{\end{trivlist}}

\renewcommand*{\proofname}{Solution}

\newenvironment{aw}
  {\begin{adjustwidth}{2em}{0em}}
  {\end{adjustwidth}}

\DeclareMathOperator{\lcm}{lcm}


\begin{document}
 
% --------------------------------------------------------------
%
%                         Start here
%
% --------------------------------------------------------------
 
\title{Chapter 2: The Integers' Solutions} % replace with the problem you are writing up
\author{Saif Mohammed} % replace with your name
\maketitle
\begin{exercise}{3}
Prove that the set of all linear combinations of \(a\) and \(b\) are precisely the multiples of \(\gcd(a, b)\).
\end{exercise}

\begin{proof}
Let $d=\gcd(a,b)$ and $c=ax+by$ for $c,x,y\in \mathbb{Z}$. We prove $c$ is a multiple of $d$.

Let $a=da'$ and $b=db'$. So $c=ax+by=d(a'x+b'y)$. Hence, $d \mid c$.

We now prove the converse: if $c$ is a multiple of $d$, then it is a linear combination of $a$ and $b$.

Let $c=dc'$. We know $d$ is the least positive linear combination of $a$ and $b$. So
\begin{align*}
    & d = am+bn & \left[\text{for }m, n\in \mathbb{Z} \right]\\
    \implies & dc' = amc'+bnc' \\
    \implies & c = a(mc')+b(nc').
\end{align*}
Hence, we proved $c$ is a linear combination of $a$ and $b$.
\end{proof}

\begin{exercise}{4}
Two numbers are said to be relatively prime if their \(\gcd\) is 1. Prove that \(a\) and \(b\) are relatively prime if and only if every integer can be written as a linear combination of \(a\) and \(b\).
\end{exercise}

\begin{proof}
Suppose $a$ and $b$ are relatively prime. We know that $ax+by=\gcd(a,b)=1$ for some $x, y\in \mathbb{Z}$. As every integer is a multiple of $1$, then by \textbf{(3)}, every integer can be written as a linear combination of $a$ and $b$.

Now, we prove the the converse: if every integer can be written as a linear combination of $a$ and $b$, then $\gcd(a,b)=1$.

As $1$ is the least positive integer and we know that $\gcd(a,b)$ must be the least positive linear combination, we can conclude that $\gcd(a,b)=1$.
\end{proof}

\begin{exercise}{5}
Prove Theorem 2.6. That is, use induction to prove that if the prime \(p\) divides \(a_1 a_2 \cdots a_n\), then \(p\) divides \(a_i\), for some \(i\).
\end{exercise}

\begin{proof}
If $p\mid a_1$, then the statement is trivially true. Now suppose if $p\mid a_1 a_2 \cdots a_k$, then $p\mid a_i$ for some $i$. Now for $p\mid a_1 a_2 \cdots a_{k+1}$, if we bring out a $a_{k+1}$, then there can happen two cases.

\textbf{Case 1:} $p\mid a_{k+1}$. If so, then the statement is obviously true.

\textbf{Case 2:} $p\nmid a_{k+1}$. Is so, then we can say $p\mid a_1 a_2 \cdots a_k$. And by the induction hypothesis, there is$a_i$ for some $i$ such that $p\mid a_i$.

Hence, the theorem is proved.

[Note: In each case, any $a_n$ not necessarily the same with other $a_n$'s. So, $a_1$ of the base case, $a_1, a_2, \ldots ,a_k$ of the induction hypothesis, and $a_1, a_2, \ldots ,a_k$ of the inductive step are not necessarily are the same numbers.]
\end{proof}

\begin{exercise}{7}
(a) A natural number greater than 1 that is not prime is called \textbf{composite}. Show that for any \( n \), there is a run of \( n \) consecutive composite numbers. \textit{Hint:} Think factorial.

(b) Therefore, there is a string of 5 consecutive composite numbers starting where?
\end{exercise}

\begin{proof}
(a) For any $n$, consider the numbers $(n+1)!+a$, where $2\leq a \leq n+1$. There are $n+1-2+1=n$ numbers, all are composite. Because, here, $a$ is also a factor of $(n+1)!$, so we write:
\begin{align*}
(n+1)!+a & = 1\cdot 2\cdots (a-1)\cdot a\cdot (a+1) \cdots (n+1) + a \\
& = a(1\cdot 2\cdots (a-1) \cdot (a+1)\cdot (a+2)\cdots (n+1)).
\end{align*}
It implies that the numbers can be written as $m=bc$, where $b\neq \pm 1$ and $c\neq \pm 1$. Hence, they are not prime.

(b) Starting from $(5+1)!+2=722$.
\end{proof}

\begin{exercise}{9}
    Notice that \(\gcd(30, 50) = 5 \gcd(6, 10) = 5 \cdot 2\). In fact, this is always true; prove that if \(a \neq 0\), then \(\gcd(ab, ac) = a \cdot \gcd(b, c)\).
\end{exercise}

\begin{proof}
    $\gcd(ab,ac)= abx+acy = a(bx+cy)$ for some $x, y\in \mathbb{Z}$. Now $\gcd(b,c)$ is multiple of any linear combination of $b$ and $c$, so
    $$
    \gcd(b,c) \mid bx + cy \implies a\cdot \gcd(b,c) \mid a(bx+cy) \implies a\cdot \gcd(b,c) \mid \gcd(ab,ac).
    $$
    Again, for some $x', y' \in \mathbb{Z}$  $\gcd(b,c) = bx'+cy' \implies a \cdot \gcd(b,c) \implies abx'+acy'$. As $\gcd(ab,ac)$ is multiple of any linear combination of $ab$ and $ac$, so
    \[
    \gcd(ab,ac) \mid abx'+acy' \implies \gcd(ab,bc) \mid a\cdot \gcd(b,c)
    \]
    Because $a\cdot \gcd(b,c) \mid \gcd(ab,ac)$ and $\gcd(ab,bc) \mid a\cdot \gcd(b,c)$, it can be concluded that $\gcd(ab,ac)=a\cdot \gcd(b,c)$.
\end{proof}

\begin{exercise}{10}
Suppose that two integers \(a\) and \(b\) have been factored into primes as follows:
\[
a = p_1^{n_1} p_2^{n_2} \cdots p_r^{n_r}
\]
and
\[
b = p_1^{m_1} p_2^{m_2} \cdots p_r^{m_r},
\]
where the \(p_i\)'s are primes, and the exponents \(m_i\) and \(n_i\) are non-negative integers. It is the case that
\[
\gcd(a, b) = p_1^{s_1} p_2^{s_2} \cdots p_r^{s_r},
\]
where \(s_i\) is the smaller of \(n_i\) and \(m_i\). Show this with \(a = 360 = 2^3 3^2 5^1\) and \(b = 900 = 2^2 3^2 5^2\). Now prove this fact in general.
\end{exercise}

\begin{proof}
For $r=1$, $a=p_1^{n_1}$ and $b=p_1^{m_1}$, where WLOG, $n_1 > m_1=s_1$. We can rewrite $a=p_1^{s_1}p_1^{n_1-s_1}$ and so $\gcd(a,b)=p_1^{s_1}\cdot \gcd(p_1^{n_1-s_1}, 1)=p_1^{s_1}$.

Let up to $r=k$ this holds. Now for $r=k+1$, there are two cases. WLOG, $n_{k+1}>m_{k+1}=s_{k+1}$.
\begin{align*}
    \gcd(a,b) & = p_{k+1}^{s_{k+1}} \cdot \gcd(p_1^{n_1} p_2^{n_2} \cdots p_k^{n_k},
    p_1^{m_1} p_2^{m_2} \cdots p_k^{m_k} p_{k+1}^{m_{k+1}-s_{k+1}}) & \\
    & = p_{k+1}^{s_{k+1}} \cdot \gcd(p_1^{n_1} p_2^{n_2} \cdots p_k^{n_k},    p_1^{m_1} p_2^{m_2} \cdots p_k^{m_k}) & \text{[no common divisor exists between } \\
    & &\text{a prime and other primes.]} \\
    & = p_1^{s_1} p_2^{s_2} \cdots p_k^{s_k} p_{k+1}^{s_{k+1}} & \text{[by induction hypothesis.]}
\end{align*}
\end{proof}

\begin{exercise}{11}
The least common multiple of natural numbers \(a\) and \(b\) is the smallest positive common multiple of \(a\) and \(b\). That is, if \(m\) is the least common multiple of \(a\) and \(b\), then \(a \mid m\) and \(b \mid m\), and if \(a \mid n\) and \(b \mid n\) then \(n \geq m\). We will write \(\text{lcm}(a, b)\) for the least common multiple of \(a\) and \(b\). Find \(\text{lcm}(20, 114)\) and \(\text{lcm}(14, 45)\). Can you find a formula for the lcm of the type given for the gcd in the previous exercise?
\end{exercise}

\begin{proof}
If $a=p_1^{n_1} p_2^{n_2} \cdots p_r^{n_r}$ and $b = p_1^{m_1} p_2^{m_2} \cdots p_r^{m_r}$, then we claim that
\[
\lcm(a,b) = p_1^{l_1} p_2^{l_2} \cdots p_r^{l_r}
\]
where $l_i$ is the largest of $n_i$ and $m_i$. 

Now we prove that. Let $c=p_1^{l_1} p_2^{l_2} \cdots p_r^{l_r}$. As $l_i$ is the largest of $n_i$ and $m_i$, it is always larger than or equal to the corresponding power of factorization of either $a$ or $b$. So $a\mid c$ and $b\mid c$.

Now let's take another common multiple of $a$ and $b$, $d=p_1^{w_1} p_2^{w_2} \cdots p_r^{w_r}$. So
\[
a \mid d \implies p_i^{n_i} \mid p_i^{w_i} \implies n_i \mid w_i \implies n_i \leq w_i
\]
and 
\[
b \mid d \implies p_i^{m_i} \mid p_i^{w_i} \implies m_i \mid w_i \implies m_i \leq w_i
\]
It follows that $w_i$ is larger than or equal to the the largest of $n_i$ and $m_i$. That is, $w_i \geq l_i$. Hence, $c \mid d \implies c\leq d$. Thus, we can conclude $c$ is indeed the $\lcm$ of $a$ and $b$.
\end{proof}

\begin{exercise}{12}
Show that if \( \gcd(a, b) = 1 \), then \( \lcm(a, b) = ab \). In general, show that \[ \lcm(a, b) = \frac{ab}{\gcd(a, b)}. \]
\end{exercise}

\begin{proof}
For $a=p_1^{n_1} p_2^{n_2} \cdots p_r^{n_r}$ and $b = p_1^{m_1} p_2^{m_2} \cdots p_r^{m_r}$, we know,
\begin{align}
& \lcm(a,b) = p_1^{l_1} p_2^{l_2} \cdots p_r^{l_r} & \text{[$l_i$ is the largest of $m_i$ and $n_i$.]} \\
\text{and \quad} & \gcd(a, b) = p_1^{s_1} p_2^{s_2} \cdots p_r^{s_r} & \text{[$s_i$ is smallest of $m_i$ and $n_i$.]}
\end{align}

Let $m_i \geq n_i$ for some $i \in \mathbb{N}$. So $l_i=m_i$ and $s_i=n_i$. Hence, $p_i^{l_i}\cdot p_i^{s_i} = p_i^{m_i}\cdot p_i^{n_i}$. Similarly, we can show that for $m_i\leq n_i$, $p_i^{l_i}\cdot p_i^{s_i} = p_i^{m_i}\cdot p_i^{n_i}$. It follows that if we multiply (1) and (2), then $p_i^{m_i}$ and $p_i^{n_i}$ both exists for any $i\in \mathbb{N}$ as factors of the product. By rearranging them, we can write
\begin{gather*}
\lcm(a,b)\cdot \gcd(a,b)=p_1^{n_1} p_2^{n_2} \cdots p_r^{n_r}\cdot p_1^{m_1} p_2^{m_2} \cdots p_r^{m_r}\\
\implies \lcm(a,b)\cdot \gcd(a,b) = ab \\
\implies \lcm(a, b) = \frac{ab}{\gcd(a, b)}.
\end{gather*}
\end{proof}

\begin{exercise}{13}
Prove that if \( m \) is a common multiple of both \( a \) and \( b \), then \( \lcm(a, b) \mid m \).
\end{exercise}

\begin{proof}
Let \( l = \lcm(a, b) \). Let's assume that \( l \nmid m \).  So, \( m = l \cdot k + r \), where \( k \in \mathbb{N} \) and \( 0 < r < \mid l \mid \).  Here, \( a \mid m \) and \( b \mid m \) and also \( a \mid l \) and \( b \mid l \), so \( a \mid r \) and \( b \mid r \). 

So we see that \( r \) is a common multiple of \( a \) and \( b \), which is less than \( l \), but this is not possible because $l$ is the least common multiple of $a$ and $b$. Therefore, \( l \mid m \implies \lcm(a, b) \mid m \).
\end{proof}














% --------------------------------------------------------------
%     You don't have to mess with anything below this line.
% --------------------------------------------------------------
 
\end{document}